\documentclass{article}
\pagenumbering{gobble} % No need for page No. as only one page is there
\renewcommand{\thesubsection}{\thesection.\alph{subsection}}
\usepackage[margin=1.0in]{geometry}
\usepackage{url} % My bibTex sources contain urls, so had to use this package
\usepackage[super,square]{natbib} % WikiPedia like citations are cooler
\title{Viral Marketing: Influential nodes}
\author{
	Ujjawal Garg \\
	San Jos\'{e} State University \\
	ujjawal.garg@sjsu.edu $\bullet$ (408)-752-6034
	}
\date{Mar 22, 2018}
\usepackage{filecontents}
\begin{filecontents}{proposal.bib}
@article{richardson,
  title={Mining Knowledge-Sharing Sites for Viral Marketing},
  author={Matthew Richardson, Pedro Domingos},
  year={2002}
}
@article{kempe,
  title={Maximising the Spread of Influence through a Social Network},
  author={David Kempe, Jon Kleinberg, Ev´a Tardos},
  year={2003}
}
@article{wang,
  title={Scalable influence maximization for prevalent viral marketing in large-scale social networks},
  author={Wei Chen, Chi Chiu Wang, Yajun Wang},
  year={2010}
}
@article{chen,
  title={Identifying influential nodes in complex networks},
  author={Duanbing Chen, Linyuan Lü, Ming-Sheng Shang, Yi-Cheng Zhang, Tao Zhou},
  year={2012}
}
@article{liu,
  title={On the Shoulders of Giants: Incremental Influence Maximisation in Evolving Social Networks},
  author={Xiaodong Liu, Xiangke Liao, Shanshan Li, Jingying Zhang, Lisong Shao, Chenlin Huang, Liquan Xiao},
  year={2015}
}
\end{filecontents}




\begin{document}
\maketitle

\begin{abstract}
\noindent % Huh? I did not like the indent here
Finding a set of influential nodes in a graph is an interesting problem in the field of viral marketing. For this project, I will explore\footnote{Other interesting papers: ~\cite{kempe} ~\cite{liu} ~\cite{wang} ~\cite{richardson}} the Susceptible Infected Recovered (SIR) model introduced by ~\cite{chen}. In this model, they propose a semi-local centrality measure as a tradeoff between the low-relevant degree centrality and other time-consuming measures.
\end{abstract}

\section*{Datasets}
\begin{enumerate}
\item Coauthorships in network science: co-authorship network of scientists working on network theory and experiment, as compiled by M. Newman in May 2006. http://www-personal.umich.edu/{\textasciitilde}mejn/netdata/netscience.zip
\item ca-HepPh: Collaboration network of Arxiv High Energy Physics. https://snap.stanford.edu/data/ca-HepPh.html
\item E-mail network URV: List of edges of the network of e-mail interchanges between members of the Univeristy Rovira i Virgili (Tarragona).  dhttp://deim.urv.cat/~netdatasci/data/email.zip
\end{enumerate}






\bibliographystyle{plainurl}
\bibliography{proposal}
\end{document}

